\chapter{Planung der Unterweisung}
% Please add the following required packages to your document preamble:
% \usepackage{booktabs}
% \usepackage{multirow}
\begin{table}[]
	\centering
	\begin{tabular}{@{}llll@{}}
		\toprule
		\textbf{Phase}       & \textbf{Lernschritt} & \textbf{Kernpunkt}                        & \textbf{Begründung}      \\ \midrule
		\multirow{4}{*}{\textit{Eröffnung}} &
		Begrüßung &
		Angenehme Lernatmospähre herstellen, Etwas Small-Talk (nicht zu persönlich) &
		Abbau möglicher Hemmungen gegenüber dem Ausbilder \\
		&
		Thema\&Ziel &
		Kurze Erläuterung des Themas und Beschreibung des Lernziels; den Sinn der Unterweisung kurz umreisen bzgl. der Anwendung in der Abteilung &
		Transparenz, Überblick, Sinnhaftigkeit erkennen \\
		&
		Motivation &
		Azubi hat n. d. Unterweisung einen Wissensvorsprung (Anerkennung durch das Umfeld, extrin.); Ausbilder brennt für dieses Thema (Vorbildfunktion, intrin.) &
		Trägt zur Lernbereitschaft bei \\
		& Vorkenntnisse        & Durch offene Fragen Vorkenntnisse finden; & Anknüpfungspunkte finden \\
		\textit{Erarbeitung} &                      &                                           &                          \\
		&                      &                                           &                          \\
		\multirow{4}{*}{\textit{Schluss}} &
		Erfolgskontrolle &
		Dazu Kapitel bla &
		Überprüfung des Erlernten; Erkennung von möglichen Lücken \\
		&
		Ausblick &
		Nächstes Thema: Weiteres Clustering-Verfahren (dbscan) + Vergleich der beiden Verfahren &
		Überblick, Transperanz und Neugierde wecken \\
		&
		Berichtsheft &
		Hinweis auf die Pflicht diese Unterweisung einzutragen; Kontroll-Termin ankündigen &
		Reflexion des Erlernten; Zulassungsvoraussetzung Abschlussprüfung \\
		& Verabschiedung       &                                           & Positives Ende          
	\end{tabular}
	\caption{Zeitlicher und fachlicher Unterweisungsplan}
	\label{tab:Unterweisungsplan}
\end{table}
