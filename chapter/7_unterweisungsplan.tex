\chapter{Planung der Unterweisung}
% Please add the following required packages to your document preamble:
% \usepackage{booktabs}
% \usepackage{multirow}
% \usepackage{longtable}
% Note: It may be necessary to compile the document several times to get a multi-page table to line up properly
\begin{longtable}[c]{@{}lp{3.5cm}p{4.5cm}p{4.0cm}@{}}
	\toprule
	\textbf{Phase} &
	\textbf{Lernschritt} &
	\textbf{Kernpunkt} &
	\textbf{Begründung} \\* \midrule
	\endfirsthead
	%
	\multicolumn{4}{c}%
	{{\bfseries Tabelle \thetable\ von vorheriger Seite fortgesetzt.}} \\
	\toprule
	\textbf{Phase} &
	\textbf{Lernschritt} &
	\textbf{Kernpunkt} &
	\textbf{Begründung} \\* \midrule
	\endhead
	%
	\multirow{4}{*}{\textit{Eröffnung}} &
	Begrüßung &
	Angenehme Lernatmosphäre herstellen; etwas Small-Talk (nicht zu persönlich) &
	Abbau möglicher Hemmungen gegenüber dem Ausbilder \\* \cmidrule(l){2-4} 
	&
	Thema\&Ziel &
	Kurze Erläuterung des Themas und Beschreibung des Lernziels; den Sinn der Unterweisung kurz umreißen bzgl. der Anwendung in der Abteilung &
	Transparenz, Überblick, Sinnhaftigkeit erkennen \\* \cmidrule(l){2-4} 
	&
	Motivation &
	Azubi hat nach der Unterweisung einen Wissensvorsprung (Anerkennung durch das Umfeld, extrin.); Ausbilder brennt für dieses Thema (Vorbildfunktion, intrin.) &
	Trägt zur Lernbereitschaft bei \\* \cmidrule(l){2-4} 
	&
	Vorkenntnisse &
	Durch offene Fragen Vorkenntnisse finden (Was kannst du mir zum Thema Clusterverfahren erzählen?) &
	Anknüpfungspunkte finden \\* \midrule
	\multirow{9}{*}{\textit{Erarbeitung}} &
	\multirow{2}{*}{Einstieg} &
	Was ist ein Cluster? Was stellst du dir darunter vor? &
	\multirow{9}{*}{offene Fragen} \\* \cmidrule(lr){3-3}
	&
	&
	Welche Eigenschaften könnten genutzt werden, um die Cluster zu bilden? &
	\\* \cmidrule(lr){2-3}
	&
	\multirow{3}{*}{Erab. Algo} &
	Wie könnte der Algorithmus vorgehen, um die oben genannten Eigenschaft zur Einteilung zu nutzen? &
	\\* \cmidrule(lr){3-3}
	&
	&
	Welche Voraussetzungen müssen gegeben sein, damit der Algorithmus funktioniert? &
	\\* \cmidrule(lr){3-3}
	&
	&
	Wie kannst du das Simulationswerkzeug nutzen, um den Algorithmus zu testen? &
	\\* \cmidrule(lr){2-3}
	&
	\multirow{4}{*}{Abschluss} &
	Welche Anwendungsfälle könntest du dir vorstellen? &
	\\* \cmidrule(lr){3-3}
	&
	&
	Welche Vorteile kannst du dir bei dem Algorithmus vorstellen? Welche Nachteile? &
	\\* \cmidrule(lr){3-3}
	&
	&
	Wie kann die Güte des Algorithmus gemessen werden? Wie stellst du dir das vor? &
	\\* \cmidrule(lr){3-3}
	&
	&
	Wie kannst du das Gelernte auf deine Aufgaben in der Abteilung übertragen? &
	\\* \midrule
	\multirow{4}{*}{\textit{Schluss}} &
	Erfolgskontrolle &
	Siehe dazu Kapitel XKAPX &
	Überprüfung des Erlernten; Erkennung von möglichen Lücken \\* \cmidrule(l){2-4} 
	&
	Ausblick &
	Nächstes Thema: Weiteres Clustering-Verfahren (dbscan) + Vergleich der beiden Verfahren &
	Überblick, Transparenz und Neugierde wecken \\* \cmidrule(l){2-4} 
	&
	Berichtsheft &
	Hinweis auf die Pflicht diese Unterweisung einzutragen; Kontroll-Termin ankündigen &
	Reflexion des Erlernten; Zulassungsvoraussetzung Abschlussprüfung \\* \cmidrule(l){2-4} 
	&
	Verabschiedung &
	&
	Positives Ende \\* \bottomrule
	\caption{Zeitlicher und fachlicher Unterweisungsplan}
	\label{tab:Unterweisungsplan}\\
\end{longtable}