\chapter{Handlungskompetenz}
Die Vermittlung der beruflichen Handlungskompetenz ist Ziel einer jeden Ausbildung: \enquote{Die Berufsausbildung hat die für die Ausübung einer qualifizierten beruflichen Tätigkeit in einer sich wandelnden Arbeitswelt notwendigen beruflichen Fertigkeiten, Kenntnisse und Fähigkeiten (berufliche Handlungsfähigkeit) in einem geordneten Ausbildungsgang zu vermitteln. Sie hat ferner den Erwerb der erforderlichen Berufserfahrungen zu ermöglichen.}\autocite[][§1\,III BBiG]{berufsbildungsgesetz_bbig_bbig_nodate} 
\par
Die Individual-, Methoden- und Sozialkompetenz bilden die Schlüsselqualifikationen der Auszubildenden. 

\section{Fachkompetenz}
Die Fachkompetenz beschreibt die Fähigkeit zur selbständigen Lösung komplexer beruflicher Aufgaben. Dies sind die Fähigkeiten und Kenntnisse die der Ausbildungsrahmenplan vorgibt. Die Vermittlung dieser ist Bestandteil der Ausbildung. 
\par
In dieser Unterweisung soll ein Beitrag zur Erlernung der Fähigkeiten bzw. der Kenntnisse \enquote{mathematische Vorhersagemodelle anwenden} und \enquote{Werkzeuge zur Mustererkennung und zur Modellgenerierung nutzen} erbracht werden.

\section{Individualkompetenz}
Dies ist Fähigkeit mit Engagement und Ausdauer die übertragenen Aufgaben auszuführen. Unter die Individualkompetenz fallen Kritikfähigkeit, Kreativität, Aufgeschlossenheit,  Lernbereitschaft, Belastbarkeit und Motivation.
\par
In der Unterweisung werden die Kreativität, Lernbereitschaft und Motivation direkt angesprochen. Die Kreativität wird benötigt, um eine eigene Lösung für die Cluster zu finden, bevor der Algorithmus die optimale Lösung berechnet. Die Lernbereitschaft und Motivation sind zwingende Bestandteile einer erfolgreicher Unterweisung. 

\section{Methodenkompetenz}
Dies beschreibt die Fähigkeit zur selbstständigen Aneignung neuer Fertigkeiten und Kenntnisse: dabei stehen die Fähigkeiten der Informationsbeschaffung, die Präsentation, die konzeptionelle Kompetenz, die Planungs- bzw. Organisationsfähigkeit und das Zeit- bzw. Selbstmanagement im Fokus.
\par
Hier lernt die Auszubildende Sabrina Dengel durch das fragend-entwickelnde Lerngespräch Cluster anhand eines definierten Vorgehens zu erkennen und dies Computer-gestützt durchzuführen.

\section{Sozialkompetenz}
Dies ist die Fähigkeit mit Menschen erfolgreich auszukommen, umzugehen und sich dabei zu behaupten und zu entwickeln. Die wichtigsten Kompetenzen sind: Organisationstalent, Verantwortungsbewusstsein, Teamfähigkeit, Belastbarkeit und Kommunikationsfähigkeit. 
\par
Während der Unterweisung wird die Kommunikationsfähigkeit sowie die Zielstrebigkeit geschult. 
