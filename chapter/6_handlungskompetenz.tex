\chapter{Handlungskompetenz}
Die Vermittlung der beruflichen Handlungskompetenz ist Ziel einer jeden Ausbildung: \enquote{Die Berufsausbildung hat die für die Ausübung einer qualifizierten beruflichen Tätigkeit in einer sich wandelnden Arbeitswelt notwendigen beruflichen Fertigkeiten, Kenntnisse und Fähigkeiten (berufliche Handlungsfähigkeit) in einem geordneten Ausbildungsgang zu vermitteln. Sie hat ferner den Erwerb der erforderlichen Berufserfahrungen zu ermöglichen.}\autocite[][§1\,III BBiG]{berufsbildungsgesetz_bbig_bbig_nodate} 
\par
Die Individual-, Methoden- und Sozialkompetenz bilden die Schlüsselqualifikationen des Auszubildenden. 

\section{Fachkompetenz}
Die Fachkompetenz beschreibt die Fähigkeit zur selbständigen Lösung komplexer beruflicher Aufgaben. Dies sind die Fähigkeiten und Kenntnisse die der Ausbildungsrahmenplan vorgibt. Die Vermittlung dieser ist Bestandteil der Ausbildung. 
\par

\section{Individualkompetenz}

\section{Methodenkompetenz}

\section{Sozialkompetenz}
