\hyphenation{Ent-wick-lungs-ge-mäß-heit}
% ---------------------------------------------
\chapter{Pädagogische bzw. didaktische Prinzipien}
In diesem Kapitel werden die einzelnen didaktischen Prinzipien kurz erläutert und auf die vorliegende Unterweisung übertragen. Es ist naheliegend, dass alle genannten pädagogischen Prinzipien in die Unterweisung einfließen, um einen bestmöglichen Erfolg zu unterstützen. Die allgemeinen Grundsätze für die didaktischen Überlegungen gelten unbeschränkt: 

\begin{itemize}
	\item vom Leichtem zum Schweren,
	\item vom Einfachen zum Zusammengesetzten, 
	\item vom Nahen zum Entfernten, 
	\item vom Allgemeinen zum Speziellen und 
	\item vom Konkreten zum Abstrakten.
	
\end{itemize}

\section{Prinzip der Anschaulichkeit}
Anschauung bezeichnet das bewusste, eindringliche, absichtliche und allseitige Erfassen eines Gegenstandes mit möglichst allen Sinnen. Ein solches allseitiges Aufnehmen eines Gegenstandes ist die Grundlage für den weiteren geistigen Prozess einer Unterweisung.\autocite[vgl.][S.\,161ff.]{schroder_lernen_2010} Allgemein gilt der Grundsatz: Anschauen $\Rightarrow$ Denken $\Rightarrow$ Anwenden.
\par
Um dieses Prinzip umzusetzen, bringt der/die Ausbilder/-in ein Informationsplakat mit dem zu vermittelnden Algorithmus mit. Auch eine anschaulich Simulation des Ablaufs des Algorithmus wird mitgebracht. Die Auszubildende kann damit diesen anschaulich nachvollziehen. Sie bekommt die Unterlagen nach der Unterweisung zur Verfügung gestellt, um eine Hilfestellung für die Aufgaben der Erfolgskontrolle zu haben.

\section{Prinzip der Aktivität}
Bei diesem Prinzip kann die/der Ausbilder/-in sich den Tätigkeitsdrang der Jugendlichen zunutze machen: Aktive und selbstständige Mitarbeitende sind für jeden Betrieb unerlässlich. Auch kann die demokratische Gesellschaft ohne aktive Mitarbeitende und kritisch denkende Staatsbürger/-innen nicht bestehen.
\par
Durch das fragend-entwickelnde Lehrgespräch ist \Azubi dazu angehalten kritisch und aktiv mitzudenken, damit ist sie aktiviert und eingebunden. Auch soll sie selbstständig im Lehrgespräch das mitgebrachte Schaubild ergänzen. Dabei zieht sie selbst Rückschlüsse durch das Wissen, das sie sich in der Vergangenheit angeeignet hat. Schließlich versucht \Azubi das Wissen zu übertragen. 

\section{Prinzip der Praxisnähe}
Die Auszubildenden, die geeignet für den Ausbildungsberuf sind, sind daran interessiert sinnvolle Arbeit zu leisten, d.\,h., sie möchten etwas lernen, das in der beruflichen Praxis verwendbar ist. Übertragen bedeutet das, es ist darauf zu achten nur aktuelle und praxisnahe Fertigkeiten, Kenntnisse und Fähigkeiten zu vermitteln. 
\par
Die Exploration bzw. die automatische Wissensgenerierung mittels Computersystem ist heutzutage nicht mehr wegzudenken. Die Unternehmen besitzen immer mehr Daten ihrer Kunden, ihres Unternehmens und ihrer Konkurrenten.\autocite[vgl.][]{noauthor_hochschule_nodate}\autocite[vgl.][]{noauthor_volkswagen_nodate} Diese Datenflut muss mit mathematischen Methoden unter Zuhilfenahme von Computersystem aufbereitet werden, da Mitarbeitende ohne Hilfe diese große Datenmenge nicht bewältigen können. Maschinelles Lernen beschreibt dieses Vorgehen. Die \ac{SVI} entwickelt Systeme im Bereich des maschinellen Lernens für ihre Kundin die \ac{SV}. Damit entspringt das Thema der Unterweisung direkt aus dem operativen Geschäftsfeld des Ausbildungsbetriebes. 

\section{Prinzip der Entwicklungsgemäßheit}
Dieses Prinzip beschreibt, dass die körperliche und/oder geistliche Überforderung des Auszubildenden unbedingt zu vermeiden ist. Weiterhin ist zu beachten, dass die Art und Weise wie ein/-e Jugendliche/-r denkt oder fühlt sich stark von der eines Erwachsenen unterscheiden kann. Übertragen bedeutet das: Die/der Ausbilder/-in hat auf den individuellen Entwicklungsstand der Auszubildenden einzugehen. 
\par
In der vorliegenden Unterweisung wird darauf geachtet, dass die Sprache und die Intensität der Wissensvermittlung an die Auszubildende angepasst wird. Die/der Ausbilder/-in achtet drauf, dass Annalena Schmidt an ihrem Kompetenz- und Interessenniveau abgeholt wird. Damit wird einer Über- oder Unterforderung entgegengewirkt. Wichtig hierbei ist die Anpassung der Sprache und des Anspruches an die Auszubildenden.

\section{Prinzip der Erfolgssicherung}
Dieses Prinzip gehört zu den obersten Leitsätzen einer jeden Unterweisung: Die/der Ausbilder/-in möchte, dass das vermittelte Wissen beim Auszubilden fest verankert wird, damit es im weiteren Verlauf der Ausbildung und im späteren Berufsalltag verwendet werden kann. 
\par
Die Erfolgssicherung wird ausführlich in Kapitel \vref{kap3:erfolg} beschrieben. Eine weitere Ausführung bedarf es aus der Sicht des Autors nicht.

\section{Prinzip der Verknüpfung}
Hier wird explizit die Wissensverknüpfung in den Mittelpunkt gestellt. Es sollen Anknüpfungspunkte für den Auszubildenden geschaffen werden, an denen sie/er sich festhalten kann, um das neu erlernte Wissen in Verknüpfung zum bereits Gelernten zu bringen. Die Anknüpfungspunkte können aus dem Betrieb oder dem Alltag kommen. Hier fließt das Prinzip der didaktischen Reduktion \enquote{vom Einfach zum Schweren und vom Bekannten zum Unbekannten} ein.
\par
Annalena Schmidt hat in vorhergehenden Abteilungen einige Anknüpfungspunkte zum Thema der Unterweisung sammeln können. So hat sie auch in der Berufsschule eine Einführung in den Themenkomplex des maschinellen Lernens erhalten. In der vorliegenden Unterweisung wird Annalena Schmidt ein weiteres Verfahren der Clusteranalyse kennen lernen. 

