\hyphenation{Ent-wick-lungs-ge-mäß-heit}
% ---------------------------------------------
\chapter{Pädagogische bzw. didaktische Prinzipien}
In diesem Kapitel werden die einzelnen didaktischen Prinzipien kurz erläutert und auf die vorliegende Unterweisung übertragen. Es ist naheliegend, dass alle genannten pädagogischen Prinzipien in die Unterweisung einfließen, um einen bestmöglichen Erfolg zu unterstützen. Die allgemeinen Grundsätze für die didaktischen Überlegungen gelten unbeschränkt: 

\begin{itemize}
	\item vom Leichtem zum Schweren,
	\item vom Einfachen zum Zusammengesetzten, 
	\item vom Nahen zum Entfernten, 
	\item vom Allgemeinen zum Speziellen und 
	\item vom Konkreten zum Abstrakten.
	
\end{itemize}

\section{Prinzip der Anschaulichkeit}

\section{Prinzip der Aktivität}

\section{Prinzip der Praxisnähe}

\section{Prinzip der Entwicklungsgemäßheit}

\section{Prinzip der Erfolgssicherung}

\section{Prinzip der Verknüpfung}
