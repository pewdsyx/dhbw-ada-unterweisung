\chapter{Lernziele, -bereiche und -kontrolle}

\section{Lernziele}
Kurz erläutern, was die Lernziele können etc. Was machen die?
\subsection{Richtlernziel}
Die Richtlernziele sind der \ac{FIAusbV} zu entnehmen: Gemäß §3 Absatz 1 FIAusbV\autocite[][§3 I FIAusbV]{bundesminister_fur_wirtschaft_und_energie_verordnung_2020} sind mindestens die im Ausbildungsrahmenplan genannten Fertigkeiten, Kenntnisse und Fähigkeiten Gegenstand der Ausbildung. Diese Ziele sind als \enquote{Teil des Berufsbildes} im Ausbildungsrahmenplan beschrieben. 
\par
Die vorliegende Unterweisung thematisiert das Richtlernziel des Abschnitts D \enquote{Nutzen der Daten zur Optimierung von Arbeits- und Geschäftsprozessen sowie zur Optimierung digitaler Geschäftsmodelle}\autocite[vgl.][§4 V Nr.\,3 FIAusbV]{bundesminister_fur_wirtschaft_und_energie_verordnung_2020}. Dieses Richtlernziel richtet sich an die Fachrichtung \enquote{Daten- und Prozessanalyse} und ist deswegen den berufsprofilgebenden Fertigkeiten, Kenntnisse und Fähigkeiten dieser Fachrichtung zu geordnet.

\subsection{Groblernziel}
Das Groblernziel wird durch die Beschreibung der zu vermittelnden Fertigkeiten, Kenntnisse und Fähigkeiten des Ausbildungsrahmenplan illustriert. Dieses Ziel konkretisiert welche Fertigkeiten, Kenntnisse und Fähigkeiten der Auszubildende nach dem Abschluss der Ausbildung beherrschen soll. Jedoch ist das Groblernziel zu unspezifisch formuliert, um es direkt zur Erfolgskontrolle zu nutzen. 
\par
Ziel der vorliegenden Unterweisung ist es, einen Beitrag zur Vermittlung der Groblernziele Nr.\,4f \enquote{mathematische Vorhersagemodelle anwenden} und Nr.\,4g \enquote{Werkzeuge zur Mustererkennung und Modellgenerierung nutzen} zu leisten. Diese Ziele sind dem Auszubildenden ab dem 19. bis zum 36. Monat zu vermitteln. Da sich die beiden Ziele ergänzen, sollen beide in die Entwicklung des Feinlernziels der Unterweisung einfließen. 
 
\subsection{Feinlernziel}
Muss ich mir selbst überlegen 

\section{Lernbereiche}
Hier werden die einzelnen Lernbereiche angesprochen, jedoch ist der psychomotorische Bereich nicht genannt, da diese Unterweisung diesen Lernbereich nicht ansprechen wird.

\subsection{Kognitiver Bereich}

\subsection{Affektiver Bereich}

%\subsection{Psychomotorischer Bereich}

\section{Erfolgskontrolle}


