\chapter{Lernziele, -bereiche und -kontrolle}

\section{Lernziele}


 

\subsection{Richtlernziel}


\subsection{Groblernziel}

 
\subsection{Feinlernziel}
 

\begin{table}[h!]
	\centering
	
	\begin{tabular}{@{}cp{8.0cm}l@{}}
		\toprule
		\textbf{Nummer} & \textbf{Feinlernziel} & \textbf{Stufe} \\ \midrule
		1 & Die allgemeine Vorgehensweise nennen, um ein Cluster zu erzeugen & Reproduzieren\\
		2 & Die Idee des Algorithmus \enquote{k-means} nennen & Reproduzieren \\
		3 & Einzelne Schritte des Algorithmus nennen, fehlende erkennen und auswerten & Reorganisation \\
%		4 & Die benötigten Werkzeuge zur Mustererkennung und Modellgenerierung nennen und fehlende erkennen & Reorganisation \\ 
		4 & Gute und schlechte Cluster erkennen & Reorganisation \\
		5 & Selbstständig neue Daten mittels des Algorithmus gruppieren & Transfer \\ 
		6 & Daten gruppieren, die nicht mittels des erlernten Algorithmus einteilbar sind & Kreativität\\ 

		\bottomrule
	\end{tabular}

	\caption{Feinlernziele der Unterweisung}
	\label{tab:lernziele}
\end{table}


\section{Lernbereiche}


\subsection{Kognitiver Bereich}


\subsection{Affektiver Bereich}


%\subsection{Psychomotorischer Bereich}

\section{Erfolgskontrolle}\label{kap3:erfolg}



