\chapter{Lernziele, -bereiche und -kontrolle}

\section{Lernziele}
Grundsätzlich ist das Lernen eine Verhaltensänderung, die 

\begin{itemize}
	\item durch Versuch, Irrtum und zufälligen Erfolg;
	\item durch Nachahmung oder 
	\item durch Einsicht
\end{itemize}

bewirkt wird. Belohnung eines gezeigten Verhaltens wirken als positiver Verstärker. Kurz definiert ist Lernen \enquote{die Veränderung einer Verhaltensweise durch Erfahrung}.
\par
Im Kern der Unterweisung geht es um das Lernen. Deswegen ist den Lernzielen besondere Aufmerksamkeit zu schenken. Lernziele sollen möglichst genau angeben welche Verhaltensweisen in welchem Maße und unter welchen Bedingungen geändert werden sollen. 

\subsection{Richtlernziel}
Die Richtlernziele sind der \ac{FIAusbV} zu entnehmen: Gemäß §3 Absatz 1 FIAusbV\autocite[][§3 I FIAusbV]{bundesminister_fur_wirtschaft_und_energie_verordnung_2020} sind mindestens die im Ausbildungsrahmenplan genannten Fertigkeiten, Kenntnisse und Fähigkeiten Gegenstand der Ausbildung. Diese Ziele sind als \enquote{Teil des Berufsbildes} im Ausbildungsrahmenplan beschrieben. 
\par
Die vorliegende Unterweisung thematisiert das Richtlernziel des Abschnitts D \enquote{Nutzen der Daten zur Optimierung von Arbeits- und Geschäftsprozessen sowie zur Optimierung digitaler Geschäftsmodelle}\autocite[vgl.][§4 V Nr.\,3 FIAusbV]{bundesminister_fur_wirtschaft_und_energie_verordnung_2020}. Dieses Richtlernziel richtet sich an die Fachrichtung \enquote{Daten- und Prozessanalyse} und ist deswegen den berufsprofilgebenden Fertigkeiten, Kenntnisse und Fähigkeiten dieser Fachrichtung zu geordnet.

\subsection{Groblernziel}
Das Groblernziel wird durch die Beschreibung der zu vermittelnden Fertigkeiten, Kenntnisse und Fähigkeiten des Ausbildungsrahmenplan illustriert. Dieses Ziel konkretisiert welche Fertigkeiten, Kenntnisse und Fähigkeiten der Auszubildende nach dem Abschluss der Ausbildung beherrschen soll. Jedoch ist das Groblernziel zu unspezifisch formuliert, um es direkt zur Erfolgskontrolle zu nutzen. 
\par
Ziel der vorliegenden Unterweisung ist es, einen Beitrag zur Vermittlung der Groblernziele Nr.\,4f \enquote{mathematische Vorhersagemodelle anwenden} und Nr.\,4g \enquote{Werkzeuge zur Mustererkennung und Modellgenerierung nutzen} zu leisten. Diese Ziele sind dem Auszubildenden ab dem 19. bis zum 36. Monat zu vermitteln. Da sich die beiden Ziele ergänzen, sollen beide in die Entwicklung des Feinlernziels der Unterweisung einfließen. 
 
\subsection{Feinlernziel}
Das Feinlernziel soll operationalisiert sein: Die genaue Zielsetzung beschreibt eine exakte Erläuterung des Lernziels mit allen Einzelheiten, die gut überprüfbar sein müssen. Daraus folgt, dass das Lernziel dann operationalisiert ist, wenn überprüfbare Verhaltensweisen festgelegt sind, anhand der Auszubildende geprüft werden kann. 
\par
Lernziele ...
%TODO: Lernziele festlegen 

\section{Lernbereiche}
Hier werden die einzelnen Lernbereiche angesprochen, jedoch ist der psychomotorische Bereich nicht genannt, da diese Unterweisung diesen Lernbereich nicht ansprechen wird.

\subsection{Kognitiver Bereich}

\subsection{Affektiver Bereich}

%\subsection{Psychomotorischer Bereich}

\section{Erfolgskontrolle}
Es werden im Anschluss der Unterweisung Kontrollfragen gestellt, die zeigen, ob das Gelernte verstanden wurde. Auch bekommt die Auszubildende Übungsaufgaben und eine kurze Zusammenfassung ausgehändigt, um die Unterweisung selbstständig nochmals nachvollziehen zu können. Die Übungsaufgaben soll Sabrina innerhalb von zwei Wochen während ihrer Arbeitszeit erledigen und dem Ausbilder vorlegen. Die Aufgaben sind so gewählt, dass die Auszubildende weder über- noch unterfordert wird.
\par
Um das Gelernte zu festigen und den Erfolg zu kontrollieren, wird Sabrina im Anschluss an die Unterweisung folgende Informationen erhalten: Ihr wird eine Ansprechpartnerin genannt bei der sie sich melden soll, um das Gelernte auf anderen Daten in Zusammenarbeit mit der Ansprechpartnerin weiter zu üben. Schließlich wird Sabrina die nächsten vier Wochen die Ansprechpartnerin begleiten und nach ihren Fähigkeiten, Kenntnissen und Fertigkeiten unterstützen. Der Ausbilder informiert die Ansprechpartnerin über das Vorhaben und Auch kontrolliert, dass die sie die Auszubildende weder über- noch unterfordert. 


