\chapter{Motivation}

\paragraph{intrinsische Motivation}
Die intrinsische Motivation ist das innere, aus sich selbst entstehende Inzentiv eines Menschen. Bestimmte Tätigkeiten werden durch die intrinsisch motivierte Person ihres selbst Willens ausgeübt, da sie Spaß machen, sinnvoll bzw. herausfordernd oder interessant sind. Wichtig ist, dass diese Art der Motivation nicht durch Belohnungen entsteht oder konditioniert wird. Der Ausbilder kann nur durch die Funktion des Vorbildes diese Motivation ansprechen, in dem sie/er für eine Tätigkeit brennt bzw. diese total spannend und toll findet. Dies kann dann auf den Auszubildenden überspringen. 
\par
Der Ausbilder brennt für das Thema sehr und hofft dieses Gefühl durch sein Verhalten auf die Auszubildende Sabrina Dengel zu übertragen. Er hofft, dass sie nach der Unterweisung genau so viel Interesse verspürt. Wenn der Ausbilder nur einen kleinen Teil seiner Begeisterung auf seine Auszubildende übertragen durch die Funktion des Vorbilds, ist das Motivieren von der Auszubildenden geglückt. 

\paragraph{extrinsische Motivation}
Diese Form der Motivation ist durch äußere Reize beeinflusst. Extrinsische Motivationsquellen sind u.\,a. der Wunsch nach Belohnung (Bezahlung der Arbeit) oder die Vermeidung einer Bestrafung (schlechte Prüfungsergebnisse). Die treibende Kraft ist die Aussicht auf Geld, Anerkennung oder Vermeidung der Strafe.
\par
