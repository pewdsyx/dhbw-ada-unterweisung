\chapter{Motivation}

\paragraph{intrinsische Motivation}
Die intrinsische Motivation ist das innere, aus sich selbst entstehende Inzentiv eines Menschen. Bestimmte Tätigkeiten werden durch die intrinsisch motivierte Person ihres selbst Willens ausgeübt, da sie Spaß machen, sinnvoll bzw. herausfordernd oder interessant sind. Wichtig ist, dass diese Art der Motivation nicht durch Belohnungen entsteht oder konditioniert wird. Der Ausbilder kann nur durch die Funktion des Vorbildes diese Motivation ansprechen, in dem sie/er für eine Tätigkeit brennt bzw. diese total spannend und toll findet. Dies kann dann auf den Auszubildenden überspringen. 
\par
Bezug auf das Thema der Unterweisung.
\paragraph{extrinsische Motivation}