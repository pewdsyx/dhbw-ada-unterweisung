\chapter[Forschungsfrage 3]{Welche besonderen sicherheitstechnischen Aspekte muss ein solcher Prozess im Bereich der Versicherung erfüllen?} \label{ff3}
Diese Kapitel ... \par
Informations- und Kommunikationssysteme sind in der heutigen Gesellschaft von elementarer Bedeutung -- sie spielen eine immer größer werdende Rolle. Der Innovationsgrad in der Informationstechnik ist konstant hoch und deswegen sind folgende Bereiche ständiger Weiterentwicklung unterlegen: steigende Vernetzung der Bevölkerung, IT-Verbreitung und Durchdringung, verschwinden der Netzgrenzen, kürze Angriffszyklen auf wichtige Infrastruktur, höhere Interaktivität von Anwendungen und die Verantwortung der Benutzer eines IT-Systems.\autocite[vgl.][S.2f.]{bundesamt_fur_sicherheit_in_der_informationstechnik_bsi_it-grundschutz-kompendium_2020}

\section{Sicherheitstechnische Anforderungen an den Betrieb einer Anwendung}
Informationen sind elementarer Bestandteil der heutigen Welt -- diese sind von sehr hohem Wert für Unternehmen und Behörden. Die meisten Geschäftsprozesse, die im heutigen Prozessablauf einer Organisation verankert sind, funktionieren nicht ohne IT-Unterstützung. Somit ist die Informationstechnologie elementarer Bestandteil jedes Unternehmens. Deswegen ist ein zuverlässiges System mit entsprechender Soft- und Hardware unerlässlich. Es muss darauf geachtet werden, dass die Informationen, die auf diesen System verteilt sind, ausreichend gut geschützt sind, damit es nicht zu einer Bedrohungslage kommt. Unzureichend geschützte Systeme stellen ein sehr hohes Risiko dar. \enquote{Dabei ist ein vernünftiger Informationsschutz ebenso wie eine Grundsicherung der IT schon mit verhältnismäßig geringen Mitteln zu erreichen. Die verarbeiteten Daten und Informationen müssen adäquat geschützt, Sicherheitsmaßnahmen sorgfältig geplant, umgesetzt und kontrolliert werden. Hierbei ist es aber wichtig, sich nicht nur auf die Sicherheit von IT-Systemen zu konzentrieren, da Informationssicherheit ganzheitlich betrachtet werden muss. Sie	hängt auch stark von infrastrukturellen, organisatorischen und personellen Rahmenbedingungen ab. }\autocite[][S.1]{bundesamt_fur_sicherheit_in_der_informationstechnik_bsi_it-grundschutz-kompendium_2020} Die Mängel in der IT-Sicherheit führen meist zu folgenden drei Kategorien von Problemen\autocite[vgl.][S.1ff.]{bundesamt_fur_sicherheit_in_der_informationstechnik_bsi_it-grundschutz-kompendium_2020}: 

\begin{itemize}
	\item Verlust der Verfügbarkeit
	\item Verlust der Vertraulichkeit
	\item Verlust der Integrität
\end{itemize}

Der Verlust der Verfügbarkeit eines IT-Systems fällt \ac{i.d.R.} sofort auf, da meist Aufgaben ohne diese Informationen nicht weitergeführt werden können. Meist fällt dies in dem Verlust der Funktionen eines Systems auf. Die Vertraulichkeit von personenbezogenen Daten ist ein bestehendes Grundrecht jedes Bürgers beziehungsweise jedes Kunden. Dies ist in verschiedenen Gesetzen wie auch Verordnung geregelt. Diese Daten müssen geschützt werden, da jedes Konkurrenzunternehmen Interesse an den Daten des Unternehmens hat. \enquote{Gefälschte oder verfälschte Daten können beispielsweise zu Fehlbuchungen, falschen Lieferungen oder fehlerhaften Produkten führen. Auch der Verlust der Authentizität (Echtheit und Überprüfbarkeit) hat, als ein Teilbereich der Integrität, eine hohe Bedeutung: Daten werden beispielsweise einer falschen Person zugeordnet. So können Zahlungsanweisungen oder Bestellungen zulasten einer dritten Person verarbeitet werden, ungesicherte digitale Willenserklärungen können falschen Personen zugerechnet werden, die digitale Identität wird	gefälscht.}\autocite[][S.1]{bundesamt_fur_sicherheit_in_der_informationstechnik_bsi_it-grundschutz-kompendium_2020}

\subsection{IT-Sicherheit: Grundnorm ISO 27001}

\subsection{IT-Grundschutz-Katalog}

\subsection{\ac{VAIT}}

\section{Beschaffung von \enquote{open/closed source}-Software}

\section{Konzept zur Implementierung der Sicherheitsanforderungen}

\section{Ergebnis der Forschungsfrage drei}


