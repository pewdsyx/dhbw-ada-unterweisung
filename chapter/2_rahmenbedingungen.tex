\chapter{Rahmenbedingungen}

\section{Auszubildende}
Die Auszubildende Sabrina Dengel begann ihre Ausbildung im September 2017 und befindet sich nun in ihrem Abschlussjahr. Sie ist somit im 34. Monat ihrer Ausbildung. 
Schulabschluss
persönliches Eignungsprofil
Integration in den Betrieb 
Wichtige Vorkenntnisse zum Thema
Alter 
Besondere Fähigkeiten 
Monat überarbeiten ... 

\section{Ausbildungsbetrieb}
Der Ausbildungsbetrieb \ac{SVI} ist eine Tochtergesellschaft der \ac{SV}, die die IT-Dienstleitungen für ihren Mutterkonzern sowie die \ac{SVS} übernimmt. Die Gesellschaften gehören dem S-Finanzbund an. Die \ac{SVI} ist ein mittelständiges Unternehmen mit dem Firmensitz Mannheim und ungefähr 450 Mitarbeitenden\autocite[vgl.][]{sv_informatik_gmbg_uber_2020} an fünf Standorten in Deutschland. Die Standorte Mannheim, Dresden, Kassel, Stuttgart und Wiesbaden sind im Geschäftsgebiet der \ac{SV} und \ac{SVS} verteilt. \enquote{Unseren Kunden bieten wir ein \enquote{Rund-um-Sorglos-Paket}[sic!]: Von der Beratung, über Konzepte bis hin zur produktiven Anwendung. Und das alles auf Basis moderner Infrastrukturen und Plattformen.}\autocite{sv_informatik_gmbg_uber_2020}
\par
Die Ausbildung der Fachinformatiker*innen erfolgt an allen Standorten. Ziel der Ausbildung ist es, die Handlungskompetenz der Auszubildenden zu fördern und sie bestmöglich auf den Einsatz an den verschiedenen Standorten vorzubereiten. Deswegen sind die Ausbildungsstationen auf alle Standorte verteilt.

\section{Ausbilder/-in}

\section{Lernort}

\section{Unterweisungszeitpunkt und -dauer}

\section{Unterweisungsmethode}

\section{Lehr- und Arbeitsmittel}


