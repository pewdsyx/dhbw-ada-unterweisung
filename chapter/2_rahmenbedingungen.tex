\chapter{Rahmenbedingungen}

\section{Auszubildende}
Die Auszubildende \Azubi begann ihre Ausbildung im September 2018 -- somit befindet sie sich im 22. Monat ihrer Ausbildung. Sabrina hat eine sehr gute mittlere Reife und ist 19 Jahre alt. Sie hat im Rahmen der Ausbildung einige Abteilungen durchlaufen und kennt dadurch die Wertschöpfungskette der \ac{SVI}. Viel wichtiger ist jedoch, dass sie durch den Einsatz in verschiedenen Abteilungen viele Mitarbeitenden kennt und positiv bekannt im Unternehmen ist. Ihre persönliche Eignung hat den Betrieb bei der Einstellung von Sabrina überzeugt: Denn durch ihre ständige Neugierde, ihre Fähigkeit sich zu fokussieren, ihr Wunsch etwas zu lernen und ihr Verlangen nach sinnhafter Arbeit führen sie mit guten bis sehr guten Leistungen durch die Ausbildung. Bevor sie in die Abteilung \enquote{Geschäftsanalytik} kam, war sie in diversen Abteilungen, die sich mit Datenaufbereitung, -validierung und -vorbereitung beschäftigt haben. Damit hat sie ein Teil der Fertigkeiten, Kenntnisse und Fähigkeiten aus §4 Absatz 5 Nummer 3 \ac{FIAusbV} erfolgreich erlernt. Diese bilden die Grundlagen, um in dieser Unterweisung Vorwissen zu haben. 

\section{Ausbildungsbetrieb}
Der Ausbildungsbetrieb \ac{SVI} ist eine Tochtergesellschaft der \ac{SV}, die die IT-Dienstleitungen für ihren Mutterkonzern sowie die \ac{SVS} übernimmt. Die Gesellschaften gehören dem S-Finanzbund an. Die \ac{SVI} ist ein mittelständiges Unternehmen mit dem Firmensitz Mannheim und ungefähr 450 Mitarbeitenden\autocite[vgl.][]{sv_informatik_gmbg_uber_2020} an fünf Standorten in Deutschland. Die Standorte Mannheim, Dresden, Kassel, Stuttgart und Wiesbaden sind im Geschäftsgebiet der \ac{SV} und \ac{SVS} verteilt. \enquote{Unseren Kunden bieten wir ein \enquote{Rund-um-Sorglos-Paket}[sic!]: Von der Beratung, über Konzepte bis hin zur produktiven Anwendung. Und das alles auf Basis moderner Infrastrukturen und Plattformen.}\autocite{sv_informatik_gmbg_uber_2020}
\par
Die Ausbildung der Fachinformatiker*innen erfolgt an allen Standorten. Ziel der Ausbildung ist es, die Handlungskompetenz der Auszubildenden zu fördern und sie bestmöglich auf den Einsatz an den verschiedenen Standorten vorzubereiten. Deswegen sind die Ausbildungsstationen auf alle Standorte verteilt.

\section{Ausbilder/-in}
Der Ausbilder Yves Staudenmaier hat ein erfolgreich abgeschlossenes Studium der Wirtschaftsinformatik mit Schwerpunkt maschinellem Lernen. Er arbeitet seit dem 01.~Juni 2015 bei der \ac{SVI} und bekleidet aktuell die Stelle eines \enquote{Senior Developer} im Bereich maschinellem Lernen. Er ist fachlich gemäß §30 BBiG geeignet, da Yves ein abgeschlossenes Hochschulstudium absolviert hat und fünf Jahre in diesem Beruf praktisch tätig ist. Des Weiteren ist Herr Staudenmaier persönlich gemäß §29 BBiG geeignet, da er Jugendliche beschäftigen darf und nicht gegen das \ac{BBiG} oder auf Grund des \ac{BBiG} erlassenen Vorschriften oder Bestimmungen wiederholt oder schwer verstoßen hat. Somit erfüllt er die Voraussetzung der Eignung eines Ausbilders gemäß §28 BBiG. Außerdem hat er im Rahmen seines Studium des Veranstaltung zum Erlangen des Ada-Scheins erfolgreich besucht. Der Ausbilder pflegt einen guten Umgang mit den Auszubildenden in der Rolle des Coachs, Mentors und Erzieher. Er ist bedacht ein gutes Betriebsklima aufrecht zu halten, um ein positives Lernklima für seine Auszubildenden zu bewahren.

\section{Lernort}
Da der Ausbilder nicht über ein eigenes Büro verfügt, hat er ein Besprechungsraum für die Unterweisung reserviert. Er ist sich darüber bewusst, dass das Großraumbüro kein geeigneter Ort ist, um eine Unterweisung durchzuführen, da keine ruhige und lernfördernde Atmosphäre entstehen kann. Der Ausbilder präpariert den großzügigen Besprechungsraum, sodass keine ungewollten Ablenkungsmöglichkeiten für den Auszubildenden und den Ausbilder vorhanden sind: Das Telefon im Raum wird deaktiviert und der Raum jeweils 20 Minuten vor und nach der Unterweisung blockiert, damit kein Zeitdruck während dieser entsteht. Auch ist der Ausbilder 25 Minuten vor dem Eintreffen des Auszubildenden im Besprechungsraum, um alles vorzubereiten. Der Raum ist dem Auszubildenden bekannt und mit großen Fenstern mit Blick in den Innenhof des Gebäudes ausgestattet, sodass viel Tageslicht eindringen kann. Durch die großen Fenster kann für eine angenehme Lüftung und Temperatur gesorgt werden, falls es doch zu warm werden sollte verfügt das Gebäude über eine Klimaanlage. Die Möblierung ist angemessen und in freundlichen Farben gehalten. Der Raum ist abgeschottet gegenüber Lärm. 

\section{Unterweisungszeitpunkt und -dauer}
Der Ausbilder ist sich der Tatsache des menschlichen Biorhythmus und dessen unterschiedliche Ausprägung bei verschiedenen Menschen bewusst. Er weiß aus früheren Gesprächen mit der Auszubildenden Sabrina Dengel, dass sie gern sehr früh morgens anfängt zu arbeiten und in den frühen Stunden ihres Arbeitstages am leistungsfähigsten ist. Auf Grund dieses Gesprächs wird die Unterweisung morgens um 9 Uhr angesetzt. Auch der Ausbilder ist, wie Frau Dengel, meist früh am Arbeitsplatz. Die Unterweisung wird maximal 1,5 Stunden dauern. Dieser Zeitraum teilt sich in folgende Bestandteile auf: die Begrüßung, ein kurzer Small-Talk, die Unterweisung, die Erfolgskontrolle, die Zeitreserve und die Verabschiedung.   

\section{Unterweisungsmethode}
Lehrgespräch. (Befindet sich in der modifizierten 4-Stufen-Methode)

\section{Lehr- und Arbeitsmittel}
Als Lehr- und Arbeitsmittel werden folgende Materialien bereitgestellt: 
%TODO: Nochmal anpassen!
\begin{itemize}
	\item Laptop mit installierter Software, die für die Unterweisung benötigt wird,
	\item Schreibutensilien, wie Papier und Stifte, 
	\item ein kleines Informationsplakat, dass das Vorgehen des Algorithmus zeigt.
\end{itemize}


