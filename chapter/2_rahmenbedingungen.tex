\chapter{Rahmenbedingungen}

\section{Auszubildende}
Die Auszubildende \Azubi begann ihre Ausbildung im September 2018 -- somit befindet sie sich im 22. Monat ihrer Ausbildung. Annalena hat eine sehr gute mittlere Reife und ist 19 Jahre alt. Sie hat im Rahmen der Ausbildung einige Abteilungen durchlaufen und kennt dadurch die Wertschöpfungskette der \ac{SVI}. Viel wichtiger ist jedoch, dass sie durch den Einsatz in verschiedenen Abteilungen viele Mitarbeitende kennt und im Unternehmen positiv bekannt ist. Die persönliche Eignung hat den Betrieb bei der Einstellung von Annalena überzeugt: ihre ständige Neugierde, ihre Fähigkeit sich zu fokussieren, ihr Wunsch etwas zu lernen und ihr Verlangen nach sinnhafter Arbeit führten sie mit guten bis sehr guten Leistungen durch die Ausbildung. Bevor sie in die Abteilung \enquote{Geschäftsanalytik} kam, war sie in diversen Abteilungen, die sich mit Datenaufbereitung, -validierung und -vorbereitung beschäftigt haben. Damit hatte sie ein Teil der Fertigkeiten, Kenntnisse und Fähigkeiten aus §4 Absatz 5 Nummer 3 \ac{FIAusbV} erfolgreich erworben. Dieses Vorwissen bildet die Grundlage für diese Unterweisen.

\section{Ausbildungsbetrieb}
Der Ausbildungsbetrieb \ac{SVI} ist eine Tochtergesellschaft der \ac{SV}, die die IT-Dienstleitungen für ihren Mutterkonzern sowie die \ac{SVS} übernimmt. Die Gesellschaften gehören dem S-Finanzbund an. Bei der \ac{SVI} handelt es sich um ein mittelständiges Unternehmen mit dem Firmensitz Mannheim und ungefähr 450 Mitarbeitenden\autocite[vgl.][]{sv_informatik_gmbg_uber_2020} an fünf Standorten in Deutschland. Die Standorte Mannheim, Dresden, Kassel, Stuttgart und Wiesbaden sind im Geschäftsgebiet der \ac{SV} und \ac{SVS} verteilt. \enquote{Unseren Kunden bieten wir ein \enquote{Rund-um-Sorglos-Paket}[sic!]: Von der Beratung, über Konzepte bis hin zur produktiven Anwendung. Und das alles auf Basis moderner Infrastrukturen und Plattformen.}\autocite{sv_informatik_gmbg_uber_2020}
\par
Die Ausbildung der Fachinformatiker*innen erfolgt an allen Standorten. Ziel der Ausbildung ist es, die Handlungskompetenz der Auszubildenden zu fördern und sie bestmöglich auf den Einsatz an den verschiedenen Standorten vorzubereiten. Deswegen sind die Ausbildungsstationen auf alle Standorte verteilt.

\section{Ausbilder/-in}
Der Ausbilder Max Müller hat ein erfolgreich abgeschlossenes Studium in Wirtschaftsinformatik. Er arbeitet seit dem 01.~Juni 2015 bei der \ac{SVI} und bekleidet aktuell die Stelle eines \enquote{Senior Developer} im Bereich maschinellem Lernen. Herr Müller ist fachlich gemäß §30\,BBiG geeignet, da er ein Hochschulstudium absolviert hat und fünf Jahre in diesem Beruf praktisch tätig ist. Des Weiteren ist Herr Müller persönlich gemäß §29\,BBiG geeignet, da er Jugendliche beschäftigen darf und nicht gegen das \ac{BBiG} oder auf Grund des \ac{BBiG} erlassenen Vorschriften oder Bestimmungen wiederholt oder schwer verstoßen hat. Somit erfüllt er die Voraussetzung der Eignung eines Ausbilders gemäß §28\,BBiG. Außerdem hat er im Rahmen seines Studiums die Veranstaltung zum Erlangen des Ada-Scheins erfolgreich besucht. Die Aufgabe des Ausbilders ist es, einen guten Umgang mit den Auszubildenden in der Rolle des Coachs, Mentors, Erziehers und des Vorbilds zu pflegen. Er ist bedacht ein gutes Betriebsklima aufrecht zu halten, um eine positive Lernatmosphäre für seine Auszubildenden zu begünstigen.

\section{Lernort}
Da die Verfügbarkeit eines eigenen Büros nicht gegeben ist, wird für die Unterweisung ein Besprechungsraum reserviert. Zwar ist das Großraumbüro kein geeigneter Ort, um eine Unterweisung durchzuführen, da keine ruhige und lernfördernde Atmosphäre entstehen kann -- allerdings wird der großzügige Besprechungsraum so präsentiert, sodass keine ungewollten Ablenkungsmöglichkeiten für den Auszubildenden und den Ausbilder entstehen können. Das Telefon im Raum wird deaktiviert und der Raum jeweils 20 Minuten vor und nach der Unterweisung blockiert, damit kein Zeitdruck während dieser Zeit entsteht. Auch ist der Ausbilder 25 Minuten vor dem Eintreffen des Auszubildenden im Besprechungsraum, um alles vorzubereiten. Der Raum ist dem Auszubildenden bekannt und mit großen Fenstern in Blickrichtung des Innenhof des Gebäudes versehen, sodass viel Tageslicht eindringen kann. Durch die großen Fenster kann für eine angenehme Lüftung und Temperatur gesorgt werden. Falls es doch zu warm werden sollte, verfügt das Gebäude über eine Klimaanlage. Die Möblierung ist angemessen und in freundlichen Farben gehalten. Der Raum ist gegenüber Lärm abgeschottet. 

\section{Unterweisungszeitpunkt und -dauer}
Der Ausbilder ist sich der Tatsache des menschlichen Biorhythmus und dessen unterschiedliche Ausprägung bei verschiedenen Menschen bewusst. Er weiß aus früheren Gesprächen mit der Auszubildenden Annalena Schmidt, dass sie gern sehr früh morgens anfängt zu arbeiten und in den frühen Morgenstunden am leistungsfähigsten ist. Auf Grund dieses Gesprächs wird die Unterweisung für morgens um 9 Uhr angesetzt. Auch der Ausbilder ist, wie Frau Schmidt, meist früh am Arbeitsplatz. Die Unterweisung dauert maximal 20 Minuten. Dieser Zeitraum teilt sich in folgende Bestandteile auf: die Begrüßung, ein kurzer Small-Talk, die Unterweisung, die Erfolgskontrolle, die Zeitreserve und die Verabschiedung.

\section{Unterweisungsmethode}
In dieser Unterweisung wird die fragend-entwickelnde Methodik des Lehrgesprächs verwendet. Dabei wird darauf geachtet Fragen, konstruktive Einwände und sonstige Beteiligung des Auszubildenden nicht zu unterbinden. Somit wird auch der affektive Lernbereich erreicht. 

\section{Lehr- und Arbeitsmittel}
Als Lehr- und Arbeitsmittel werden folgende Materialien bereitgestellt: 
%TODO: Nochmal anpassen!
\begin{itemize}
	\item Laptop mit installierter Software, die für die Unterweisung benötigt wird,
	\item Schreibutensilien, wie Papier und Stifte, 
	\item ein kleines Informationsplakat, dass das Vorgehen des Algorithmus zeigt;
	\item Zusammenfassung des Gelernten,
	\item Übungsaufgaben und 
	\item Zugang zu den Simulationsdateien.
\end{itemize}


